\documentclass{article}
\title{Esame Java base}
\author{Andrea Balbo Mossetto \\ Davide Marietti}
% hide the date in the maketitle
\date{}

\usepackage{listings}
\usepackage{color}

\definecolor{codegreen}{rgb}{0,0.6,0}
\definecolor{codegray}{rgb}{0.5,0.5,0.5}
\definecolor{codepurple}{rgb}{0.58,0,0.82}
\definecolor{backcolour}{rgb}{0.95,0.95,0.92}

\lstdefinestyle{mystyle}{
    language=Java,
    aboveskip=5mm,
    belowskip=5mm,
    columns=flexible,
    backgroundcolor=\color{backcolour},   
    commentstyle=\color{codegreen},
    keywordstyle=\color{magenta},
    numberstyle=\tiny\color{codegray},
    stringstyle=\color{codepurple},
    basicstyle=\ttfamily\footnotesize,
    breakatwhitespace=false,         
    breaklines=true,                 
    captionpos=t,                    
    keepspaces=true,                 
    numbers=left,                    
    numbersep=5pt,                  
    showspaces=false,                
    showstringspaces=false,
    showtabs=false,                  
    tabsize=2
}

\lstset{style=mystyle}

\begin{document}
	\maketitle
	\newpage
	
	\tableofcontents
	\newpage
	
	\lstlistoflistings
	\newpage
	
	
	\section{Descrizione del lavoro}
	
	Per la realizzazione della consegna siamo partiti dalla traccia fornita dalla Professoressa, con cui venivamo invitati ad implementare uno \textit{Shelter} per animali. 
	
	Il primo passo \'e quindi consistito nell'allestimento di una classe \textit{Animale}, 
	
	\begin{lstlisting}[caption={Allestimento della classe Animale}]
		abstract class Animal {
		 	private String name, breed;
   		private int age, conditions; // condizioni di salute dell'animale adottato

			// costruttore
   		public Animal(String breed, int age, int conditions) {
      	this.breed = breed;
      	this.age = age;
      	this.conditions = conditions;
   		}

		
		// some methods here 
		}
	\end{lstlisting}
	
	
	\newpage
	
	\appendix
	\section{Codice Sorgente}
	
	\begin{lstlisting}[caption={Animal.java}]
// Definendo Animal come astratta posso usare il Binding Dynamico 
// per astrarre lo shalter come un array di tipo
// apparente Animal, ma di tipo esatto quello specifico della specie 
// che estende la classe astratta Animal.
abstract class Animal {
   private String name, breed;
   private int age, conditions; // condizioni di salute dell'animale adottato

   public Animal(String breed, int age, int conditions) {
      this.breed = breed;
      this.age = age;
      this.conditions = conditions;
   }

   // OVERLOADING: ridefinisco il costruttore della classe Animal 
   // per accettare anche l'inizializzazione del campo "name"
   public Animal(String name, String breed, int age, int conditions) {
      this.name = name;
      this.breed = breed;
      this.age = age;
      this.conditions = conditions;
   }

   // Converto l'astrazione dell'animale in una stringa. OVERWRITING: ridefinisco il metodo toString() della classe Object
   public String toString() {
      String s = "\n nome: " + name + ", razza: " + breed + ", eta': " + age + ", condizioni generali: " + conditions + "%";
      return s;
   }

   public String getName() {
      return name;
   }

   public void setName(String name) {
      this.name = name;
   }

   public String getBreed() {
      return breed;
   }

   public int getAge() {
      return age;
   }

   public String getConditions() {
      return name;
   }

   public void setConditions(int conditions) {
      this.conditions = conditions;
   }

   // Il metodo equals() fa parte dei metodi "standard" delle librerie Java, lo ridefinisco perchè mi serve definire
   // un concetto di uguaglianza tra gli animali.
   public boolean equals(Animal otherAnimal) {
      // Per definire l'uguaglianza uso i campi comuni a tutti gli animali. Tali campi rendono univoco il
      // riconoscimento di un animale ospitato nello shelter
      return
              (this.name.equalsIgnoreCase(otherAnimal.name))
                      &&
                      (this.breed.equalsIgnoreCase(otherAnimal.breed))
                      &&
                      (this.age == otherAnimal.age);
   }
}


class Dog extends Animal {

   // Subclasses add more feature alla classe che specializzano
   private String size;

   public Dog(String breed, int age, String size, int conditions) {
      super(breed, age, conditions);
      this.size = size;
   }

   public Dog(String name, String breed, int age, String size, int conditions) {
      super(name, breed, age, conditions);
      this.size = size; // NB: this esplicito
   }

   // OVERWRTING del metodo toString()
   public String toString() {
      String s = super.toString() + ", taglia: " + size + ", animale: cane"; // NB: this implicito

      return s;
   }
}


class Cat extends Animal {
   private boolean longFur;

   public Cat(String breed, int age, boolean longFur, int conditions) {
      super(breed, age, conditions);
      this.longFur = longFur;
   }

   public Cat(String name, String breed, int age, boolean longFur, int conditions) {
      super(name, breed, age, conditions);
      this.longFur = longFur;
   }

   public String toString() {
      String s = super.toString() + ", pelo lungo: " + longFur + ", animale: gatto";;

      return s;
   }
}


class Sheep extends Animal {
   private String sex;

   public Sheep(String breed, int age, String sex, int conditions) {
      super(breed, age, conditions);
      this.sex = sex;
   }

   public Sheep(String name, String breed, int age, String sex, int conditions) {
      super(name, breed, age, conditions);
      this.sex = sex;
   }

   public String toString() {
      String s = super.toString() + ", sesso: " + sex + ", animale: pecora";

      return s;
   }
}


class Donkey extends Animal {

   public Donkey(String breed, int age, int conditions) {
      super(breed, age, conditions);
   }

   public Donkey(String name, String breed, int age, int conditions) {
      super(name, breed, age, conditions);
   }

   public String toString() {
      String s = super.toString() + ", animale: asino";

      return s;
   }
}


class Piglet extends Animal {
   public Piglet(String breed, int age, int conditions) {
      super(breed, age, conditions);
   }

   public Piglet(String name, String breed, int age, int conditions) {
      super(name, breed, age, conditions);
   }

   public String toString() {
      String s = super.toString() + ", animale: maialino";

      return s;
   }
}⏎  
	\end{lstlisting}
	
\begin{lstlisting}[caption = {User.java}]
	public class User {
   private String name;
   private String surname;
   private String email;
   private String cell;

   public User(String name, String surname, String email, String cell){
      this.name = name;
      this.surname = surname;
      this.email = email;
      this.cell = cell;
   }

   public boolean adoptAnimal(AnimalShelter shelter, String name){
      return shelter.animalAdopted(name);
   }

   public void cureAnimal(Animal animal){
      animal.setConditions(100);
   }
}⏎ 
\end{lstlisting}

\begin{lstlisting}[caption={AnimalShelter.java}]
	public class AnimalShelter {

   private int nAnimals = 0; // inizialmente lo shelter non contiene animali
   private Animal[] hostedAnimals;

   // Definisco lo shelter come un array di oggetti, quindi avente una capienza limitata a maxNumAnimals
   public AnimalShelter(int maxNumAnimals) {
      hostedAnimals = new Animal[maxNumAnimals];
   }

   // Converto l'astrazione dell'AnimalShelter in una stringa. Notare la chiamata al metodo toString della classe Animal.
   public String toString() {
      String s = "Numero di animali ospitati = " + nAnimals;

      int i = 0;
      while (i < nAnimals) {
         s = s + hostedAnimals[i].toString();
         ++i;
      }

      return s;
   }

   public int getNumAnimals() {
      return nAnimals;
   }

   public int findHostedAnimalIndex(String name) {
      int i = 0;
      while (i < nAnimals) {
         if (hostedAnimals[i].getName().equalsIgnoreCase(name)) return i;
         ++i;
      }

      return nAnimals;
   }

   public boolean checkIfPresent(String name) {
      return (findHostedAnimalIndex(name) < nAnimals);
   }

   public boolean checkIfFull() {
      return (nAnimals == hostedAnimals.length);
   }

   public boolean hostAnimal(Animal animal, String name) {

      if (checkIfFull()){
         return false; // Se lo shelter è pieno il motodo fallisce
      }

      animal.setName(name);
      hostedAnimals[nAnimals] = animal;
      ++nAnimals; // aggiorno il numero degli animali presenti

      return true;
   }

   public boolean animalAdopted(String name) {
      int i = findHostedAnimalIndex(name);
      if (i == nAnimals) {
         return false; // se l'animale non esiste il motodo fallisce
      }

      --nAnimals;
      hostedAnimals[i] = hostedAnimals[nAnimals];

      return true;
   }
}⏎  
\end{lstlisting}

\begin{lstlisting}[caption={DemoAnimalShelter.java}]
public class DemoAnimalShelter {
   public static void main(String arg[]) {

      // Inizializzo un utente
      User Mila = new User("Mila", "Racca", "Mila.Racca@me.it", "+39 377 24583");
      User Piero = new User("Piero", "Gillono", "Piero.Gillono@me.it", "+39 347 58232");

      // Inizializzo lo shelter definendo il nume di posti che dispone per ospitare gli animali
      AnimalShelter shelter = new AnimalShelter(100);

      Animal dog1 = new Dog("Labrador", 6, "mid", 90);
      Animal dog2 = new Dog("Bassotto", 6, "small", 80);
      Animal cat1 = new Cat("European", 5, false,85);
      Animal cat2 = new Cat("Siamese", 2, true,33);
      Animal sheep1 = new Sheep("Irlandese", 4, "female",95);
      Animal sheep2 = new Sheep("Yorkshire", 4, "female",60);
      Animal sheep3 = new Sheep("British Columbia", 4, "male",70);
      Animal donkey1 = new Donkey("Sarda", 10,90);

      // Ospitando un animale, gli assegno un nome :)
      shelter.hostAnimal(dog1, "Buddy");
      shelter.hostAnimal(dog2, "Otto von Bismarck");
      shelter.hostAnimal(cat1, "Tom");
      shelter.hostAnimal(cat2, "Dorotea");
      shelter.hostAnimal(sheep1, "Margarita");
      shelter.hostAnimal(sheep2, "Jimmy");
      shelter.hostAnimal(sheep3, "Ed");
      shelter.hostAnimal(donkey1, "Arturo");

      System.out.println("\nSituazione dello shelter a un mese dall'apertura:\n" + shelter);

      System.out.println("\nOtto von Bismarck è ospitato nello shelter: " + shelter.checkIfPresent("Otto von Bismarck"));

      Mila.adoptAnimal(shelter, "Otto von Bismarck");

      System.out.println("Otto von Bismarck viene adottato. \nOtto von Bismarck è ospitato nello shelter: " + shelter.checkIfPresent("Otto von Bismarck"));

      System.out.println("\nIl veterinario riesce a curare i due gatti presenti in struttura.");
      Piero.cureAnimal(cat1);
      Piero.cureAnimal(cat2);

      System.out.println("\nSituazione dello shelter:\n" + shelter);
   }
}⏎   
\end{lstlisting}
	
	
\end{document}